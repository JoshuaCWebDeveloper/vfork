% Copyright 2008 2009 2010 2011 Douglas Wikstrom

% This file is part of Verificatum.

% Verificatum is free software: you can redistribute it and/or modify
% it under the terms of the GNU Lesser General Public License as
% published by the Free Software Foundation, either version 3 of the
% License, or (at your option) any later version.

% Verificatum is distributed in the hope that it will be useful, but
% WITHOUT ANY WARRANTY; without even the implied warranty of
% MERCHANTABILITY or FITNESS FOR A PARTICULAR PURPOSE.  See the GNU
% Lesser General Public License for more details.

% You should have received a copy of the GNU Lesser General Public
% License along with Verificatum.  If not, see
% <http://www.gnu.org/licenses/>.

\documentclass[11pt]{article}
\usepackage{vtm}
%\usepackage{tex4ht}


% For not wasting space in submission
\addtolength{\voffset}{-1.5cm}
\addtolength{\textheight}{3cm}

\addtolength{\hoffset}{-1cm}
\addtolength{\textwidth}{2.0cm}

%%%%%%%%%%%%%%%%%%%%%%%%%%%%%%%%%%%%%%%%%%%%%%%%%%%%%%%

\newcommand{\MARK}{$^\bigast$}


\title{{\Huge DRAFT}\\$\quad$\\User Manual for the Verificatum Mix-Net}

\author{Douglas Wikstr{\"o}m\\
KTH Stockholm, Sweden\\
\texttt{dog@csc.kth.se}}


\pagestyle{empty}

\begin{document}

%\Configure{$}{\PicMath}{\EndPicMath}{}


\maketitle

\thispagestyle{empty}

\begin{abstract}

  Verificatum, \url{http://www.verificatum.org}, is a free and open
  source implementation of an El Gamal-based mix-net. This document
  describes how to use the mix-net.

\end{abstract}

\begin{quote}

\vfill

\subsection*{Help Us Improve This Document}

This document is a draft. The most recent version of this
document can be found at:
\url{http://www.verificatum.org/verificatum/vmnu.pdf}.

Please help us improve the quality of this document by reporting
errors, omissions, and suggestions to
\texttt{dog@csc.kth.se}.


\end{quote}

\newpage

\begin{spacing}{0.9}

\setcounter{tocdepth}{1}

  \tableofcontents

\end{spacing}

\clearpage

\pagestyle{plain}
\setcounter{page}{1}

\section{Introduction}\label{sect:introduction}

The \veri mix-net~\cite{vmn} is an implementation of an \elgamal-based
re-encryption mix-net. It can be configured in many ways, but all
values have sensible defaults.

All that is needed to execute the mix-net is to run a few commands in
the right sequence. Thus, we hope that the protocol is easy to use
even for people who have no background in cryptography. The following
commands are provided.
\begin{itemize}

\item \vmni{} -- Info file generator used to generate configuration
  files for the mix-net. Some optional configuration parameters are
  outputs from the object generator \vog{} described below.

\item \vmn{} -- Mix-server executing the \veri mix-net. The execution
  is parametrized by configuration files output from \vmni{}.

\item \vog{} -- Object generator of primitive cryptographic objects
  such as hash functions, keys, and pseudo-random generators. These
  can then, using \vmni{}, be used to replace the default values.

\end{itemize}
Throughout the document we mark advanced sections by an asterix. These
are sections that target programmers or users that must use special
configurations.


\paragraph{Are you simply looking for an example?} 

\app{umexample} contains a complete description of the commands
executed by the respective operators in an execution with three
mix-servers, including how to generate demo ciphertexts using the
\vmnd{} command.

\section{Info File Generator}\label{sect:infogen}

Before the mix-net is executed, the operators of the mix-servers must
agree on a set of common parameters, generate their private and public
parameters, and share their public parameters.

\subsection{Basic Usage}

The info file generator is used in three simple steps. Below we walk
through an example with three mix-servers, where we describe the view
of the operator of the first mix-server. The other operators execute
corresponding commands.
\begin{enumerate}

\item\textbf{Agree on Common Parameters.} The operators agree on the
  name and session identifier of the execution, and the maximal number
  of ciphertexts to be processed, and then generate a stub of a
  protocol file. To do that each operator invokes \vmni{} as follows.

\vspace{0.3cm}
\begin{lstlisting}[frame=single,language=xml,
basicstyle=\tt,showstringspaces=false]
vmni -prot -sid "Session ID" -name "Swedish Election" \
     -nopart 3 -thres 2 -maxciph 10000 stub.xml
\end{lstlisting}

  The command produces a protocol info stub file \code{stub.xml}. The
  parties can then verify that they hold identical protocol info stub
  files by computing digests as described below.

  The session identifier can be used to separate multiple executions
  that logically should have the same name. In the example, the number
  of mix-servers is 3 of which 2 are needed to decrypt ciphertexts
  encrypted with the joint public key produced during the key
  generation phase. The integer 10000 gives a bound on the number of
  ciphertexts that can be input to the mix-net and indicates that
  pre-computation is used. If this option is dropped, then no
  pre-computation takes place, and calling \code{vmn -precomp}, as
  explained in \sect{mixnet}, \step{precomp}, will fail and give a
  warning.


\item\textbf{Generate Individual Info Files.} Using the protocol info
  stub file \code{stub.xml} as a starting point, each operator
  generates its own private info file and protocol info file by
  invoking \vmni{} as in the following example.

\vspace{0.3cm}
\begin{lstlisting}[frame=single,language=xml,
basicstyle=\tt,showstringspaces=false]
vmni -party -name "Mix-server 1" stub.xml \
     privInfo.xml localProtInfo.xml
\end{lstlisting}

  The command produces two files: a private info file
  \code{privInfo.xml} and an \emph{updated} protocol info file
  \code{localProtInfo.xml}. The former contains private data, e.g., the
  signature key and the local directory storing the state of the
  mix-server. The latter defines the public parameters of a party, e.g.,
  its IP address and public signature key. Each party shares its local
  protocol info file with the other mix-servers using an out-of-bound
  channel. The protocol info stub file \code{stub.xml} can now be
  deleted.

\item\textbf{Merge Protocol Info Files.}  At this point the operator
  holds three protocol info files: its local file
  \code{localProtInfo.xml}, and the local files \code{protInfo2.xml},
  and \code{protInfo3.xml} of the other parties. It merges these files
  using the following command.

\vspace{0.3cm}
\begin{lstlisting}[frame=single,language=xml,
basicstyle=\tt,showstringspaces=false]
vmni -merge localProtInfo.xml protInfo2.xml protInfo3.xml \
     protInfo.xml
\end{lstlisting}

  This produces a single \emph{joint} protocol info file
  \code{protInfo.xml} containing all the public information about common
  parameters and all parties. The output file does not depend on the
  order of the input protocol info files (see \sect{configure} for more
  information).

\end{enumerate}

\noindent
In practice, the operators could, e.g., organize a physical meeting to
which they bring their laptops and execute the above steps.


\paragraph{Computing Digests of Info Files.}

For convenience, hexadecimal encoded hash digests of files can be
computed using \vmni{} to allow all parties to check that they hold
identical protocol info files at the end. In our example, a SHA-256
digest of the joint protocol info file can be computed as follows.

\vspace{0.3cm}
\begin{lstlisting}[frame=single,language=xml,
basicstyle=\tt,showstringspaces=false]
vmni -digest protInfo.xml
\end{lstlisting}

\subsection{Execution in a Directory}\label{sect:vmniindir}

In a typical application, each operator creates a directory and
executes the above commands in this directory. For convenience, the
\vmni{} command allows the operator to drop many info file names when
executing the commands in this way, in which case the file names
default to the file names used above. A similar convention is later
used for the \vmn{} command described in \sect{mixnet}. More
precisely, the commands below are equivalent to the above.
\begin{enumerate}

\item\textbf{Agree on Common Parameters.} The following creates a
  stub info file \code{stub.xml}.

\vspace{0.3cm}
\begin{lstlisting}[frame=single,language=xml,
basicstyle=\tt,showstringspaces=false]
vmni -prot -sid "Session ID" -name "Swedish Election" \
     -nopart 3 -thres 2 -maxciph 10000
\end{lstlisting}

\item\label{step:party}\textbf{Generate Individual Info Files.} The
  following assumes
  that there is a stub file named \code{stub.xml} in the working
  directory and then creates a private info file \code{privInfo.xml}
  and a protocol info file \code{localProtInfo.xml}.

\vspace{0.3cm}
\begin{lstlisting}[frame=single,language=xml,
basicstyle=\tt,showstringspaces=false]
vmni -party -name "Mix-server 1"
\end{lstlisting}

\item\textbf{Merge Protocol Info Files.} The following creates a
  protocol info file \code{protInfo.xml}.

\vspace{0.3cm}
\begin{lstlisting}[frame=single,language=xml,
basicstyle=\tt,showstringspaces=false]
vmni -merge localProtInfo.xml protInfo2.xml protInfo3.xml
\end{lstlisting}

\end{enumerate}

\subsection{Additional Configuration Options\MARK}\label{sect:configure}

The command \vmni{} used to generate info files accepts a large number
of options which allows defining various parameters of the mix-net. In
our example we have simply used the default values, but we discuss a
few of the options below. For a complete usage description use the
following command, or generate info files as above and inspect the
resulting files (a comment is generated for each value).

\vspace{0.3cm}
\begin{lstlisting}[frame=single,language=xml,
basicstyle=\tt,showstringspaces=false]
vmni -h
\end{lstlisting}

\paragraph{Running Multiple Mix-servers on a Single Computer.}

Each mix-server allocates two ports for communication: one for its
HTTP server and one for its ``hint server''. By default these port
numbers are 8040 and 4040, and this typically works well when running
a single mix-server. However, if there is a need to run several
mix-servers on the same computer, e.g., when trying out \veri, then
the mix-servers must be assigned distinct port numbers. The
\code{-http} and \code{-httpl} options are used to define the external
and local URL's for the HTTP server. These can be distinct e.g., if
port forwarding is used. If only \code{-http} is given, then the local
port number defaults to the same value. If only \code{-httpl} is given
then the external port number remains 8040. Similarly, \code{-hint},
and \code{-hintl} options are used to define the socket address of the
hint server. Below we give an example, but \app{umexample} also
illustrates the use of these options.

\vspace{0.3cm}
\begin{lstlisting}[frame=single,language=xml,
basicstyle=\tt,showstringspaces=false]
-http http://server.example.com:8040
-hint server.example.com:4040
\end{lstlisting}

\paragraph{Secure Source of Randomness.}

The default source of randomness is the \code{/dev/urandom}
device. This is often a reasonable choice, but on machines with few
system events, this may not give sufficient entropy. The \code{-rand}
option can be used to either use a different device, e.g., a hardware
random generator mounted as a device, or a pseudo-random generator. In
the latter case, the \code{-seed} option must also be used to provide
the name of a file containing a relatively short truly random
seed. Use the \vog{} tool described in \sect{objgen} to generate a
hexadecimal-encoded instance of a random source that can be used as a
value with \code{-rand}. Below we give two examples.

\vspace{0.3cm}
\begin{lstlisting}[frame=single,language=xml,
basicstyle=\tt,showstringspaces=false]
-rand "$(vog -gen RandomDevice /dev/urandom)"
-rand "$(vog -gen PRGElGamal -fixed 2048)" -seed /dev/urandom
\end{lstlisting}


\paragraph{Mix-Net Interface.}

From an abstract point of view, the mix-net first outputs a file
containing a joint public key package with an embedded description of
the underlying group. Then it takes a file containing ciphertexts as
input and produces a file containing plaintexts as output. The format
used for these files depend on the application and can be changed
using the \code{-inter} option. The value can be one of the strings
\code{native}, \code{raw}, or \code{helios}, or the name of a subclass
of
\code{verificatum.protocol.mixnet.MixNetElGamalInterface}. \sect{inter}
describes these formats and how to implement such a subclass.


\paragraph{Roles of the Mix-servers.}

The roles of the mix-servers are not entirely symmetric. If there are
$\Mk$ mix-servers of which $\thres$ are needed to decrypt, then only
the first $\thres$ of the mix-servers take part in the shuffling, but
all take part in the decryption. The order in which the public
information of the mix-servers appear in the protocol info file
determines the roles of the mix-servers. When the local protocol info
files are merged the sections belonging to the different parties are
sorted before the result is output as a joint protocol info file. The
parties are sorted based on the \code{-srtbyrole} parameter and their
names. If two parties have the same name, then merging fails, so this
is gives an unambiguous ordering of the parties.


\section{Mix-Net}\label{sect:mixnet}

When the info files of all mix-servers have been generated the mix-net
can be executed in two (or three) simple steps.

\begin{nicebox}
  \WARNING{It is the responsibility of the user to make sure that the
    input ciphertexts are distinct and unrelated to any ciphertexts
    submitted by honest senders. Failure to do so allows a fully
    practical attack on the privacy of any sender.}
\end{nicebox}


\subsection{Basic Usage}

We complete the example from \sect{infogen} by describing the sequence
of commands executed by the operator to actually run the mix-net.
\begin{enumerate}

\item\textbf{Generate Public Key.} The operators execute the joint key
  generation phase of the protocol which outputs a joint public key to
  a file \code{publicKey} to be used by senders when computing their
  ciphertexts.
  
\vspace{0.3cm}
\begin{lstlisting}[frame=single,language=xml,
basicstyle=\tt,showstringspaces=false]
vmn -keygen privInfo.xml protInfo.xml publicKey
\end{lstlisting}

We stress that this command invokes a protocol, so all operators must
execute this command roughly at the same time.\footnote{Currently the
  implementation of the bulletin board is not robust, so there is essentially no time-out.}

  The format used to output the public key to the file
  \code{publicKey} is defined by the \emph{mix-net interface}. See
  \sect{configure} for how to choose which interface to use, and see
  \sect{inter} for details on the available interfaces and how
  implement your own.

\item\label{step:precomp}\textbf{Pre-compute (optional).} If the
  \code{-maxciph} option was used when generating the protocol info
  stub file, then pre-computation can (optionally) be executed as a
  separate step to speed up the mixing phase that follows. To start
  the pre-computation phase, the operator executes the following
  command.

\vspace{0.3cm}
\begin{lstlisting}[frame=single,language=xml,
basicstyle=\tt,showstringspaces=false]
vmn -precomp privInfo.xml protInfo.xml
\end{lstlisting}

  We stress that the the mix-servers execute a protocol during the
  pre-computation phase. Thus, all of the operators must execute this
  command roughly at the same time.

\item\textbf{Mix Ciphertexts.}  To start the mixing phase with a file
  \code{ciphertexts} containing ciphertexts, the operator uses the
  following command.

\vspace{0.3cm}
\begin{lstlisting}[frame=single,language=xml,
basicstyle=\tt,showstringspaces=false]
vmn -mix privInfo.xml protInfo.xml ciphertexts plaintexts
\end{lstlisting}

\vspace{0.2cm}
\noindent
We stress that this command invokes a protocol, so all operators must
execute this command roughly at the same time.

The formats of the files \code{ciphertexts} and \code{plaintexts} are
defined by the mix-net interface. See \sect{configure} for how to
choose which interface to use, and see \sect{inter} for details on the
available interfaces and how implement your own.

\end{enumerate}

\subsection{Resetting Securely}

If there is a fatal error during the execution, or if an operator
interrupts an execution by mistake, then all operators can execute the
following command to securely reset the states of all the mix-servers
to their states right after the end of the key generation step. Then
the options \code{-precomp} and \code{-mix} can be used to execute the
mix-net again.

\vspace{0.3cm}
\begin{lstlisting}[frame=single,language=xml,
basicstyle=\tt,showstringspaces=false]
vmn -reset privInfo.xml protInfo.xml
\end{lstlisting}

\vspace{0.2cm}
\noindent
If pre-computation was carried out for too few ciphertexts, then the
operators must manually edit their protocol info files and set the
maximal number of ciphertexts to zero. Then they can use the
\code{-mix} option to execute the mix-net without pre-computation.

Please note that resetting deletes all values computed during
pre-computation. This is required to be able to securely reset from an
arbitrary point in the protocol.


\subsection{Execution in a Directory}

If the \vmni{} command was used within a directory as explained in
\sect{vmniindir}, then the info file names can be dropped in the calls
to \vmn{} above. More precisely, the following calls can be used
instead. Each command assumes that there is a private info file named
\code{privInfo.xml} and a protocol info file \code{protInfo.xml} in
the working directory.
\begin{enumerate}

\item\textbf{Generate Public Key.} 

\begin{lstlisting}[frame=single,language=xml,
basicstyle=\tt,showstringspaces=false]
vmn -keygen publicKey
\end{lstlisting}

\item\textbf{Pre-compute (optional).}

\begin{lstlisting}[frame=single,language=xml,
basicstyle=\tt,showstringspaces=false]
vmn -precomp
\end{lstlisting}

\item\textbf{Mix Ciphertexts.}

\begin{lstlisting}[frame=single,language=xml,
basicstyle=\tt,showstringspaces=false]
vmn -mix ciphertexts plaintexts
\end{lstlisting}

\end{enumerate}

\paragraph{Securely Resetting.}

The info files can of course be dropped also when executing the reset
command.

\vspace{0.3cm}
\begin{lstlisting}[frame=single,language=xml,
basicstyle=\tt,showstringspaces=false]
vmn -reset
\end{lstlisting}


\section{Interfaces to the Mix-Net}\label{sect:inter}

The interface of a mix-net is the format used to represent: the joint
public key, the input ciphertexts, and the output plaintexts. The
representation of the joint public key also embeds a representation of
the underlying group. The \veri mix-net is configured to use a
particular interface using the \code{-inter} option to \vmni{}. As
explained in \sect{configure}, parameter to this option can be either
the short name of a built-in interface (\code{native}, \code{raw}, or
\code{helios}), or the name of a class implementing the
interface. Below we describe the built-in interfaces in terms of byte
trees and the representations of group elements defined in
\app{bytetrees} and \app{groupelems} (and in a different context in
\cite{vmnv}), and then describe how to implement a custom
interface. Here $\Gq$ denotes the underlying group, $\MSpace$ denotes
the group of plaintexts (some fixed power of $\Gq$), and
$\CSpace=\MSpace\times\MSpace$ denotes the group of ciphertexts.

\subsection{Native Interface}

The native interface converts the binary objects of the raw interface
to their hexadecimal representation and does not require the number of
ciphertexts in the input file to be available. It also decodes the
output plaintext group elements into strings according to the encoding
scheme of the underlying group. More precisely,
$\tohex{\node{\marshal{\Gq},\bt{\pk}}}$ is output on the public key
file, where $\pk$ denotes the public key in $\Gq\times\Gq$. For each
line in the file of input ciphertexts an attempt is made to interpret
it as $\tohex{\bt{\ciph}}$ for some ciphertext
$\ciph\in\CSpace$. Lines for which this fails are ignored. The array
$\mess$ of plaintext group elements in $\MSpace$ are decoded
element-wise into strings using the decoding scheme of the underlying
group. Any occurrence of a newline or carriage return character in a
string is deleted before the strings are output on file separated by
newline characters.

\subsection{Raw Interface}

The raw interface uses the internal representation of the public key
(with an embedded representation of the underlying group), the input
ciphertexts, and output plaintext group elements. More precisely, at
the end of the key generation phase, the public key file contains
$\node{\marshal{\Gq},\bt{\pk}}$, where $\pk$ is the public key in the
product group $\Gq\times\Gq$. The array $\List{0}$ of input
ciphertexts in $\CSpace$ is represented on file as $\bt{\List{0}}$ and
the array $\mess$ of output plaintext group elements in $\MSpace$ is
represented on file as $\bt{\mess}$. We stress that the output group
elements are not decoded into strings.

\subsection{Helios Interface}

The Helios interface assumes that a modular group is used and that
$\MSpace=\Gq$, i.e., that the underlying group must be an instance of
\code{verificatum.arithm.ModPGroup} and that each plaintext is a
single group elements. A modular group is used by \vmni{} by
default. The format of the public key file is best explained by an
example. Suppose that $\p=23$, $\q=11$, and that $\Gq$ is the subgroup
of order $\q$ in $\zed_{\p}^*$. Furthermore, let $\g=3$ and let
$\y=13$. Then the public key is represented as
\begin{align*}
  \code{\{"g":"3","p":"23","q":"11","y":"13"\}}\enspace.
\end{align*}
Similarly, a ciphertext $(12,16)\in\Gq\times\Gq$ is represented as
\begin{align*}
  \code{\{"alpha":"12","beta":"16"\}}
\end{align*}
and the input file of ciphertexts contains a single such line for each
ciphertext without any additional delimiters. The output file of
plaintexts is constructed exactly like in the native interface.

\subsection{Custom Interface\MARK}

The interface of the mix-net is captured by a subclass of
\code{MixNetElGamalInterface} (found in the
\code{verificatum.protocol.mixnet} package). This class requires every
subclass to implement five methods.
\begin{itemize}

\item \code{writePublicKey} -- Writes a public key to file.

\item \code{readPublicKey} -- Reads a public key from file, including
  the underlying group.
  
\item \code{writeCiphertexts} -- Writes ciphertexts to file.

\item \code{readCiphertexts} -- Reads ciphertexts from file.

\item \code{decodePlaintexts} -- Decodes plaintext group elements and
  writes the result to file.

\end{itemize}
Please consider the source of, e.g.,
\code{MixNetElGamalInterfaceNative}, for an example.



\section{Object Generator}\label{sect:objgen}

Some of the option parameters passed to \vmni{} can be complex
objects, i.e., a provably secure pseudo-random generator may be based
on a computational assumption that must be part of the encoding. The
object generator \vog{} is used to generate representations of such
objects. Before any objects are generated, the source of randomness of
the object generator must be initialized, but we postpone the
discussion of this to \sect{rndinit}.

\subsection{Listing and Generating Objects}

The main usage of \vog{} is to list all suitable subclasses of some
class specified as a valid parameter to \vmni{}, and then to generate
an instance of such a class.

\paragraph{Browse Library.}

Consider an option to \vmni{} that is parametrized by instance of a
subclass of a class \code{AbstractClass}. Then the set of subclasses
of \code{AbstractClass} that can be instantiated using \vog{} can be
listed using the following command.

\vspace{0.3cm}
\begin{lstlisting}[frame=single,language=xml,
basicstyle=\tt,showstringspaces=false]
vog -list AbstractClass
\end{lstlisting}

\vspace{0.2cm}
\noindent
For example, the source of randomness used by a protocol can be chosen
by passing an instance of a subclass of
\code{verificatum.crypto.RandomSource} to \vmni{} using the
\code{-rand} option. (Use \code{vmni -h} to find out which type of
object can be passed as a parameter with each option.) To list all
suitable subclasses, the following command can be used.

\vspace{0.3cm}
\begin{lstlisting}[frame=single,language=xml,
basicstyle=\tt,showstringspaces=false]
vog -list RandomSource
\end{lstlisting}

\vspace{0.2cm}
\noindent
As illustrated by the example it suffices to give the unqualified
class name when this is not ambiguous.

\paragraph{Generate Object.}

To generate an instance of \code{ConcreteClass} the \code{-gen} option
is used along with the name of the class, but each class requires its
own set of parameters. To determine the correct set of parameters the
following command can be used.

\vspace{0.3cm}
\begin{lstlisting}[frame=single,language=xml,
basicstyle=\tt,showstringspaces=false]
vog -gen ConcreteClass -h
\end{lstlisting}

\vspace{0.2cm}
\noindent
This prints usage information as if \code{vog -gen ConcreteClass} was
a command on its own. An instance is then generated by passing the
correct parameters, e.g., to generate an instance of
\code{HashfunctionHeuristic} that represents SHA-512, the following
command can be used.

\vspace{0.3cm}
\begin{lstlisting}[frame=single,language=xml,
basicstyle=\tt,showstringspaces=false]
vog -gen HashfunctionHeuristic SHA-512
\end{lstlisting}

\vspace{0.2cm}
\noindent
For some classes, the parameters passed to \vog{} must in turn be
generated using \vog{} itself, e.g., \code{PRGHeuristic} optionally
takes the representation of a hash function as input as illustrated in
the following.

\vspace{0.3cm}
\begin{lstlisting}[frame=single,language=xml,
basicstyle=\tt,showstringspaces=false]
vog -gen PRGHeuristic \
    "$(vog -gen HashfunctionHeuristic SHA-256)"
\end{lstlisting}

\vspace{0.2cm}
\noindent
This approach of constructing parametrized complex objects is quite
powerful. We can for example construct an instance of
\code{PRGCombiner} that combines a random device and two pseudo-random
generators using the following command.

\vspace{0.3cm}
\begin{lstlisting}[frame=single,language=xml,
basicstyle=\tt,showstringspaces=false]
vog -gen PRGCombiner                               \
    "$(vog -gen RandomDevice /dev/urandom)"        \
    "$(vog -gen PRGElGamal -fixed 1024)"           \
    "$(vog -gen PRGHeuristic                       \
       "$(vog -gen HashfunctionHeuristic SHA-256)" \
    )"
\end{lstlisting}


\subsection{Initializing the Random Source}\label{sect:rndinit}

Before \vog{} is used to generate any objects, its source of
randomness must be initialized. This is only done once. The syntax is
almost identical to the syntax to generate an instance of a subclass
of \code{RandomSource}, except that it is mandatory to provide a seed
if a pseudo-random generator is used. We give two examples. The first
example initializes the random source to be the random device
\code{/dev/urandom}. Any device can of course be used, e.g., a
hardware random generator mounted as a device. To avoid accidental
reuse of randomness, this option should \emph{never} be used with a
normal file.

\vspace{0.3cm}
\begin{lstlisting}[frame=single,language=xml,
basicstyle=\tt,showstringspaces=false]
vmn -rndinit RandomDevice /dev/urandom
\end{lstlisting}

\vspace{0.2cm}
\noindent
The second example shows how to initialize the random source as a
pseudo-random generator where the seed is read directly from
\code{/dev/urandom}.

\vspace{0.3cm}
\begin{lstlisting}[frame=single,language=xml,
basicstyle=\tt,showstringspaces=false]
vog -seed /dev/urandom -rndinit \
    PRGHeuristic "$(vog -gen HashfunctionHeuristic SHA-256)"
\end{lstlisting}


\vspace{0.2cm}
\noindent
A representation of the random source is stored in the file
\code{.verificatum\_random\_source} in the home directory of the
user. If the environment variable \code{VERIFICATUM\_RANDOM\_SOURCE}
is defined, then it is taken to be the name of a file to be used
instead. If the random source is a pseudo-random generator, i.e., a
subclass of \code{verificatum.crypto.PRG}, then the hexadecimal
encoding of its seed is stored in the file
\code{.verificatum\_random\_seed}, or if
\code{VERIFICATUM\_RANDOM\_SEED} is defined it is interpreted as a
file name to be used instead. Note that the seed is automatically
replaced with part of the output of the pseudo-random generator in
each invocation to avoid accidental reuse of the seed.


\subsection{Custom Objects\MARK}

To allow \vog{} to instantiate a custom subclass \code{CustomClass},
e.g., of \code{PGroup}, a separate subclass \code{CustomClassGen} of
\code{verificatum.ui.gen.Generator} must also be implemented. Please
note the naming convention where \code{Gen} is added as a suffix to
the class name. The generator class provides the command-line
interface to \code{CustomClass}, i.e., it prints usage information,
and interprets command-line parameters and returns an instance of
\code{CustomClass}. See source for \code{HashfunctionHeuristicGen} for
a simple example.

There are two ways to make \vog{} aware of such custom classes: (1) a
colon-separated list of classes can be provided as a parameter with
the \code{-pkgs} option, or (2) the environment variable
\code{VERIFICATUM\_VOG} can be initialized to such a list. Each class
identifies a package to be considered by \vog{}, so it suffices to
provide a single class from each package to be considered.


\section{Universally Verifiable Proof of Correctness}

By default, \vmni{} generates a protocol info stub file such that all
zero-knowledge proofs computed during the execution of \vmn{} are made
\emph{non-interactive} using the Fiat-Shamir heuristic. When \vmn{} is
executed in this mode it stores all the relevant intermediate results
along with the non-interactive zero-knowledge proofs in the
\code{roProof} subdirectory of the working directory of \vmn{}. The
contents of this directory can then be verified by anybody that has
the necessary knowledge to implement a verification algorithm. Thus,
the \veri mix-net is said to be \emph{universally verifiable}. 

\subsection{The \veri Mix-Net Verifier}

The \veri built-in verifier command \vmnv{} can handle all the
built-in mix-net interfaces. Let \code{protInfo.xml} be a protocol
info file and let \code{roProof} be a directory containing the
intermediate values and non-interactive zero-knowledge. Then the
consistency of these can be verified using the following command.

\vspace{0.3cm}
\begin{lstlisting}[frame=single,language=xml,
basicstyle=\tt,showstringspaces=false]
vmnv protInfo.xml roProof
\end{lstlisting}

\vspace{0.2cm}
\noindent
A given mix-net interface is of course used during the execution of
\vmn{} and the public key file \code{publicKey} handed to senders, the
file \code{ciphertexts} containing input ciphertexts, and the output
plaintext file \code{plaintexts} are represented using this
interface. To verify that these files correspond correctly to the
public key, input ciphertexts, and the output plaintexts stored in the
raw format within the \code{roProof} the following command can be
used (which also verifies the contents of \code{roProof}).

\vspace{0.3cm}
\begin{lstlisting}[frame=single,language=xml,
basicstyle=\tt,showstringspaces=false]
vmnv protInfo.xml roProof publicKey ciphertexts plaintexts
\end{lstlisting}

\vspace{0.2cm}
\noindent

\subsection{Independent Stand-alone Mix-Net Verifier\MARK}

Universal verifiability is of course more interesting if independent
parties implement stand-alone verifiers. These verifiers should
preferably share no code with \veri itself.

In a companion document~\cite{vmnv} targeting programmers of such
verifiers, the formats of the files in the \code{roProof} directory
and the algorithms that must be implemented are described in
detail. There is also a simple reference implementation written in
Python\cite{pvmnv} that follows the notation in~\cite{vmnv} closely.


\section{Demonstrator}

In the \code{demo/mixnet} subdirectory of the installation directory
of the \veri package contains a number of demo scripts. The following
simple command runs the demo with the default options.

\vspace{0.3cm}
\begin{lstlisting}[frame=single,language=xml,
basicstyle=\tt,showstringspaces=false]
./demo
\end{lstlisting}

\vspace{0.2cm}
\noindent
The demo can be configured to illustrate almost every option of the
mix-net. It is also easy to configure it to orchestrate an execution
on multiple computers remotely. For more information we refer the
reader to the \code{README} file and the configuration file
\code{conf} found in the \code{demo/mixnet} subdirectory.

\paragraph{Concrete Example.}

For a more concrete example, \app{umexample} contains a worked example
of the commands executed by the operators including how to generate
demo ciphertexts using the \vmnd{} command.

\section{Troubleshooting}

\begin{itemize}

\item If you get a ``Connection refused.'' error, then it could be
  that the HTTP-server resolves your hostname, e.g.,
  server.example.org to \code{127.0.0.1}, i.e., localhost. This is not
  a problem in \veri. The problem is that your server resolves your
  hostname incorrectly. Often it is possible to solve this problem by
  editing \code{/etc/hosts}. Google to figure out how this is done on
  your OS.

\item If you get an ``Invalid socket address!'' error, then it could
  be that the port number you are trying to use is already allocated
  by an other mix-server (or other server). See \sect{configure} for
  how to run multiple mix-servers on a single computer.

\end{itemize}


\section{Acknowledgments}

The suggestions of Shahram Khazaei and Gunnar Kreitz have improved
this document.


\bibliographystyle{abbrv}
\bibliography{vtm}

\appendix

\section{Commands for an Execution with Three Mix-servers}\label{sect:umexample}

\lstset{
%frame=single,
keywordstyle=\it\ttfamily,
showstringspaces=false,
language=xml,
float,
basicstyle=\tt\xmlfontsize,
}

\lstinputlisting{umexample.tex}


\section{Byte Trees}\label{sect:bytetrees}

We use a byte-oriented format to represent objects on file and to turn
them into arrays of bytes that can be input to a hash function. The
goal of this format is to be as simple as possible.

\subsection{Definition}

A byte tree is either a \emph{leaf} containing an array of bytes, or a
\emph{node} containing other byte trees. We write \leaf{$d$} for a
leaf with a byte array $d$ and we write \node{$b_1,\ldots,b_l$} for a
node with children $b_1,\ldots,b_l$. Complex byte trees are then easy
to describe.

\begin{example}\label{exam:bytetree}
  The byte tree containing the data \hex{AF}, \hex{03E1}, and
  \hex{2D52} (written in hexadecimal) in three leaves, where the first
  two leaves are siblings is
\begin{align*}
  \node{\node{\leaf{\hex{AF}}, \leaf{\hex{03E1}}}, \leaf{\hex{2D52}}}\enspace.
\end{align*}
\end{example}

\subsection{Representation as an Array of Bytes}

We use $\nbytes{k}{n}$ as a short-hand to denote the two's-complement
representation of $n$ in big endian byte order truncated to the $k$
least significant bytes. We also use hexadecimal notation for
constants, e.g., $\hex{0A}$ means $\nbytes{1}{10}$.

A byte tree is represented as an array of bytes as follows.
\begin{itemize}

\item A leaf $\leaf{d}$ is represented as the concatenation of: a
  single byte $\hex{01}$ to indicate that it is a leaf, four bytes
  $\nbytes{4}{l}$, where $l$ is the number of bytes in $d$, and the
  data bytes $d$.

\item A node $\node{b_1,\ldots,b_l}$ is represented as the
  concatenation of: a single byte \hex{00} to indicate that it is a
  node, four bytes $\nbytes{4}{l}$ representing the number of
  children, and $\bytes{b_1}\mid\bytes{b_2}\mid\cdots\mid\bytes{b_l}$,
  where $\mid$ denotes concatenation and $\bytes{b_i}$ denotes the
  representation of the byte tree $b_i$ as an array of bytes.

\end{itemize}

\begin{example}[\exam{bytetree} contd.]
  The byte tree is represented as the following array of bytes.
  \begin{align*}
    &\hex{00\cutt00\cut00\cut00\cut02}\\
    &\quad\hex{00\cutt00\cut00\cut00\cut02}\\
    &\quad\quad\hex{01\cutt00\cut00\cut00\cut01\cutt AF}\\
    &\quad\quad\hex{01\cutt00\cut00\cut00\cut02\cutt03\cut E1}\\
    &\quad\hex{01\cutt00\cut00\cut00\cut02\cutt 2D\cut 52}
  \end{align*}
\end{example}


It is a good idea to hand (an upper bound of) the expected recursive
depth to a parser to let it fail in a controlled way if the input does
not represent a valid byte tree.

ASCII strings are converted to byte trees in the
natural way, i.e., a string $s$ (an array of bytes) is converted to
$\leaf{s}$.
\begin{example}
  The string $\code{"ABCD"}$ is represented by $\leaf{\hex{65666768}}$.
\end{example}
Sometimes we store byte trees as the hexadecimal encoding of their
representation as an array of bytes. We denote by $\tohex{a}$ the
hexadecimal encoding of an array of bytes. We abuse notation and
simply write $a$ instead of $\bytes{a}$ when the context is clear,
e.g., if $\hash$ is a hash function we write $\hash(a)$ instead of
$\hash(\bytes{a})$.

\subsection{Backus-Naur Grammar.}

The above description should suffice to implement a parser for byte
trees, but for completeness we give their Backus-Naur grammar. We
denote the $n$-fold repetition of a symbol
$\langle\textit{rule}\rangle$ by $n\langle\textit{rule}\rangle$. The
grammar then consists of the following rules for
$n=0,\ldots,2^{31}-1$, where $\mid$ denotes choice.
\begin{align*}
  \langle\textit{bytetree}\rangle&::=\langle\textit{leaf}\rangle\mid\langle\textit{node}\rangle\\
  \langle\textit{leaf}\rangle&::=\hex{01}\langle\textit{uint}_n\rangle\langle\textit{data}_n\rangle\\
  \langle\textit{node}\rangle&::=\hex{00}\langle\textit{uint}_n\rangle\langle\textit{bytetrees}_n\rangle\\
  \langle \textit{uint}_n\rangle&::=\nbytes{4}{n}\\
  \langle \textit{data}_n\rangle&::=n\langle\textit{byte}\rangle\\
  \langle\textit{bytetrees}_n\rangle&::=n\langle\textit{bytetree}\rangle\\
  \langle \textit{byte}\rangle&::=\hex{00}\mid \hex{01}\mid
  \hex{02}\mid\cdots\mid \hex{FF}
\end{align*}



\section{Representations of Groups and Group
  Elements}\label{sect:groupelems}

Group elements are represented as byte trees. In this section we pin
down the details of these representations.

\begin{itemize}

\item\textbf{Modular Group.} A subgroup $\Gq$ of order $\q$ of the
  multiplicative group $\zed_{\p}^*$, where $\p>3$ is prime, with
  standard generator $\g$ is represented by the byte tree
  $\node{\bt{\p},\bt{\q},\bt{\g},\nbytes{4}{e}}$, where the integer
  $e\in\{0,1,2\}$ determines how a string is encoded into a group
  element. We denote by $\ModPGroup(b)$ the group recovered from such
  a byte tree $b$.

  We may only have $e=1$ if $\p=2\q+1$, and we may only have $e=2$ if
  $\p=t\q+1$ and the difference between the bit lengths of $\p$ and
  $\q$ is less than $2^{10}$.

  The three encoding schemes are defined as follows.
  \begin{itemize}

  \item If $e=0$, then at most 3 bytes can be encoded. Let $\hash$
    denote SHA-256.

    \begin{description}

    \item[Encode.] To encode a sequence $\mess$ of $0\leq
      b_{\mess}\leq3$ bytes into an element $a$, find an element
      $a\in\Gq$ such that the first $b_{\mess}+1$ bytes of
      $\hash(\bt{a})$ are of the form $\nbytes{1}{l}\mid\mess$ for
      some $l$ such that $l\bmod 4=b_{\mess}$.

    \item[Decode.] To decode an element $a\in\Gq$ into a message
      $\mess$, let $\nbytes{1}{l'}$ be the first byte of
      $\hash(\bt{a})$ and define $l=l'\bmod4$. Then let $\mess$ be the
      next $l$ bytes of $\hash(\bt{a})$.

    \end{description}

  \item If $e=1$, then at most $b=\lfloor(\secp-2)/8\rfloor-4$ bytes
    can be encoded, where $\secp$ is the bit length of $\p$.
    \begin{description}

    \item[Encode.] To encode a sequence $\mess$ of $0\leq
      b_{\mess}\leq b$ bytes as an element $a\in\Gq$ do as follows.
      If $b_{\mess}=0$, then set $\mess'=\hex{01}\mid\nbytes{b-1}{0}$
      and otherwise set
      $\mess'=\mess\mid\nbytes{b-b_{\mess}}{0}$. Then interpret
      $\nbytes{4}{b_{\mess}}\mid\mess'$ as a positive integer
      $k$. Then let $a$ be $k$ or $\p-k$ so that $a\in\Gq$.

    \item[Decode.] To decode an element $a\in\Gq$ to a message
      $\mess$, set $k$ equal to $a$ or $\p-a$ depending on if $a<\p-a$
      or not. Then interpret $k\bmod 2^{8(b+4)}$ as
      $\nbytes{4}{l}\mid\mess'$, where $\mess'$ is a sequence of $b$
      bytes. If $l<0$ or $l>b$, then set $l=0$. Then let $\mess$ be the
      $l$ first bytes of $\mess'$.

    \end{description}

  \item If $e=2$, then at most $b=\lfloor\secp/8\rfloor-\secp'-4$
    bytes can be encoded, where $\secp$ is the bit length of $\p$ and
    $\secp'=\lceil t/8\rceil+1$.
    \begin{description}

    \item[Encode.] To encode a sequence $\mess$ of $0\leq
      b_{\mess}\leq b$ bytes as an element $a\in\Gq$, interpret
      $\nbytes{4}{b_{\mess}}\mid\mess\mid\nbytes{b-b_{\mess}}{0}$ as
      an integer $k$. Then let $a$ be the smallest integer of the form
      $i2^{8(b+4)}+k$ in $\Gq$ for some non-negative $i$.

    \item[Decode.] To decode an element $a\in\Gq$ to a message
      $\mess$, interpret $a\bmod 2^{8(b+4)}$ as
      $\nbytes{4}{l}\mid\mess'$, where $\mess'$ is a sequence of $b$
      bytes. If $l<0$ or $l>b$, then set $l=0$. Then let $\mess$ be the
      $l$ first bytes of $\mess'$

    \end{description}

  \end{itemize}

\item\textbf{Element of Modular Group.} An element $a\in\Gq$, where
  $\Gq$ is a subgroup of order $\q$ of $\zed_{\p}^*$ for a prime $\p$
  is represented by $\leaf{\nbytes{k}{a}}$, where $a$ is identified
  with its integer representative in $[0,\p-1]$ and $k$ is the
  smallest integer such that $\p$ can be represented as
  $\nbytes{k}{\p}$.

  \begin{example}
    Let $\Gq$ be the subgroup of order $\q=131$ in
    $\zed_{263}^*$. Then $258\in\Gq$ is represented by
    \hex{01\cutt00\cut00\cut00\cut02\cutt01\cut02}.
  \end{example}

\item\textbf{Array of Group Elements.} An array $(a_1,\ldots,a_l)$ of
  group elements is represented as $\node{\bt{a_1},\ldots,\bt{a_l}}$,
  where $\bt{a_i}$ is the byte tree representation of $a_i$.

\item\textbf{Product Group Element.} An element $a=(a_1,\ldots,a_l)$
  in a product group is represented by
  $\node{\bt{a_1},\ldots,\bt{a_l}}$, where $\bt{a_i}$ is the byte tree
  representation of $a_i$. This is similar to the representation of
  product rings.

\item\textbf{Array of Product Group Elements.} An array
  $(a_1,\ldots,a_l)$ of elements in a product group, where
  $a_i=(a_{i,1},\ldots,a_{i,k})$, is represented by
  $\node{\bt{b_1},\ldots,\bt{b_k}}$, where $b_i$ is the array
  $(a_{1,i},\ldots,a_{l,i})$ and $\bt{b_i}$ is its representation as a
  byte tree. This is similar to the representation of arrays of
  elements in a product ring.


\end{itemize}

\paragraph{Marshalling Groups.}

When objects convert themselves to byte trees in \veri, they do not
store the name of the Java class of which they are instances. Thus, to
recover an object from such a representation, information about the
class must be otherwise available. Thus, a group $\Gq$ is marshalled
into a byte tree 
\begin{align*}
  \node{\leaf{\code{"PGroupClass"}},\bt{\Gq}}\enspace,
\end{align*}
where $\Gq$ is an instance of a Java class \code{PGroupClass}, and
this is denoted $\marshal{\Gq}$.



\end{document}

%%% Local Variables: 
%%% mode: latex
%%% TeX-master: t
%%% End: 
